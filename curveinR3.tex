\chapter{\texorpdfstring{$\mathbb{R}^3$}{R\textthreesuperior}中的曲线}\label{chapter_curveinR3}

本章是前后那些``抽象而恐怖''的章节之间的一个休息, 我们来讨论一下古典曲线论的内容.
一般来说$\mathbb{R}^3$中的曲线指的是一个连续或者光滑的映射$c:I\to\mathbb{R}^3$, 其中$I$是一个区间.
我们也会提到\textit{正则性}, 这是说对$t\in I$总有
\[c'(t)=((c^1)'(t),(c^2)'(t),(c^3)'(t))\neq 0\]
接下来我们假设曲线都是光滑而正则的, 然后开启我们的旅程.