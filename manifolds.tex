\setcounter{chapter}{-1}
\chapter{准备}
\section{几何学是什么?}
现代几何学源于古希腊.
在古希腊语中, ``几何学''一词为γεωμετρία \textit{(geōmetría)}, 意为测量大地.
这反映了早期几何学主要是对长度, 面积, 体积, 角度的经验性原理的收集, 主要用于满足实用性用途.
直至今日, 初步的几何学教学仍然是从对几何体的大小的直观认识开始的.
因此几何学的一个经典要件就是\emph{度量}.

Euclid所著的\textit{Elements} (汉语中通常称作《几何原本》) 是古希腊几何的代表.
他在\textit{Elements}的开篇引入了这样的一条公理:
\begin{quotation}
    {\bfseries 公理}4.\ 彼此能重合的物体是全等的.
\end{quotation}
然后第一个引用了这条公理的命题是
\begin{quotation}
    {\bfseries 命题}4.\ 如果两个三角形中, 一个的两边分别等于另一个的两边, 而且这些线段所夹的角相等.
    那么它们的底边等于底边, 这样其余的角也等于相应的角, 即那些等边所对应的角.\\
    \begin{flushright}
        (译文引自\parencite{Euclid_Elem})
    \end{flushright}
\end{quotation}
在这最原始的直觉中, ``重合''蕴含了运动的概念, 而边角的相等则蕴含了不变量的概念.
因此, 几何学的另一个经典要件就是\emph{变换与不变量}.

古希腊的几何学主要研究直线与圆锥曲线.
到了微积分发明之后, 数学家可以使用微积分的工具来研究更一般的几何体了.
Leibniz通过密切圆引入了曲线的曲率, Bernoulli与Euler研究了曲面的法曲率与测地线.
对``曲''的研究正式进入了几何学之中.
1827年, Gauss在论文\textit{Disquisitiones generales circa superficies curvas} (关于曲面的一般研究) 中证明了``Gauss绝妙定理'' (Theorema Egregium).
从此一种观念开始进入几何学: 我们可以研究抽象的几何体, 而不考虑它在欧式空间中的实现.
进而不久非欧几何便产生了.

回到几何学的两个要件上.
有了以上基础, 1854年Riemann写作了论文\textit{\"{U}ber die hypothesen welche der Geometrie zu Grunde liegen} (论奠定几何学基础的假设), 正式引入了Riemann度量与流形的概念 (之后的笔记中我们会详细解释这两个概念).
同时, 19世纪正在经历一个射影几何的复兴潮, 当时正在流行使用射影变换的方法研究射影几何.
于是在1893年F.\ Klein发表了对整个几何学的``总结性''综述\textit{Vergleichende Betrachtungen \"{u}ber neuere geometrische Forschungen} (关于近代几何学研究的比较考察).
文章中提到几何学的目的在于
\begin{quotation}
    给定一个流形和其上的变换群, 建立关于这个变换群的不变量理论.
    \begin{flushright}
        (\parencite{Klein_Erlangen}, 自译)
    \end{flushright}
\end{quotation}

自此, 经典几何学的舞台已经搭好.
但我们还有一个问题: \label{intro_sect}
\begin{pro}
    微分几何是什么?
\end{pro}
接下来我们开始慢慢搭微分几何的舞台.