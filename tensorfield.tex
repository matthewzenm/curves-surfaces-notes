\chapter{流形上的张量}
我们回到几何学, 这一章我们讨论上一章的代数工具在微分几何中的具体应用.
我们首先会定义流形$M$上的张量场, 并讨论张量场与向量场的张量积之间的关系.
其次我们会讨论流形上的Riemann度量, 从而向Riemann几何迈出微小的一步.
这一节中我们也会引出单位分解定理, 并讨论它的应用.
最后我们会讨论交错映射构成的张量场, 也就是微分形式, 我们会略微谈及它的应用.

\section{张量场}
\begin{defn}
    微分流形$M$上一点$p$的\textbf{余切空间}是这一点处切空间的对偶空间, 记为$T^*_pM:=(T_pM)^*$.
    $T^*_pM$中的元素称为\textbf{余向量}.
\end{defn}

命题~\ref{defn of dual}~说明了$T^*_pM$具有向量空间的结构, 且维数与$M$相同.
那么类似命题~\ref{diff_stru_TM}~可以证明对$p\in M$, $T^*_pM$的不交并是一个微分流形.
我们于是有定义
\begin{defn}
    设$M$是微分流形, $\displaystyle T^*M:=\coprod_{p\in M}T^*_pM$是一个$2n$维微分流形, 定义为$M$的\textbf{余切丛}.
\end{defn}

由于任意向量空间的张量积都是向量空间, 所以$\tensor^{(r,s)}(V)$也是向量空间, 于是进一步地可以定义
\begin{defn}
    设$M$是微分流形, 定义$\displaystyle T^{(r,s)}TM:=\coprod_{p\in M}\tensor^{(r,s)}(T_pM)$为$M$的$(r,s)$-张量丛.
\end{defn}

仿照向量场的定义, 我们如下定义张量场:
\begin{defn}
    流形$M$上的一个$(r,s)$-张量场$A$是一个满足$\pi\circ A=1_M$的光滑映射$A:M\to T^{(r,s)}TM$, 其中$\pi:T^{(r,s)}TM\to M$是自然投影映射.
\end{defn}

在有了全局的张量场的定义之后, 我们考虑张量场在局部, 也就是某个坐标卡下的表现.
首先我们考虑余切空间.
\begin{lem}
    设$n$维流形$M$上的函数$x^1,\cdots,x^n$定义如引理~\ref{lem_indep}~所述, 那么$x^1_*|_p,\cdots,x^n_*|_p$是
    $\displaystyle\left.\frac{\partial}{\partial x^1}\right|_p,\cdots,\left.\frac{\partial}{\partial x^n}\right|_p$的对偶基.
    我们在这里将$T_{x^i(p)}\mathbb{R}$与$\mathbb{R}$视为等同的.
\end{lem}
\begin{proof}
    对任意$f\in C^\infty(\mathbb{R})$有
    \begin{align*}
        x^i_*|_p\left(\left.\frac{\partial}{\partial x^j}\right|_p\right)(f)&=\left.\frac{\partial f\circ x^i}{\partial x^j}\right|_p\\
        &=\frac{\partial f\circ x^i\circ\varphi^{-1}}{\partial u^j}\\
        &=\frac{\partial f\circ\pi^i}{\partial u^j}\\
        &=\frac{\d f}{\d x}\delta(i,j)
    \end{align*}
    注意到$\d/\d x\in T_{x^i(p)}\mathbb{R}$等同于$1\in\mathbb{R}$, 所以上式说明了$x^1_*|_p,\cdots,x^n_*|_p$是
    $\displaystyle\left.\frac{\partial}{\partial x^1}\right|_p,\cdots,\left.\frac{\partial}{\partial x^n}\right|_p$的对偶基.
\end{proof}

\begin{prop}
    设$A$是一个$(r,s)$-张量场 \rmparen{不一定是光滑的}, 那么$A$是光滑张量场当且仅当对任意坐标卡$(U,\varphi)$, $A$都具有
    \[\sum_{\substack{i_1\cdots i_k\\j_1\cdots j_l}}A^{i_1\cdots i_k}_{j_1\cdots j_l}\frac{\partial}{\partial x^{i_1}}\otimes\cdots\otimes\frac{\partial}{\partial x^{i_k}}\otimes x^{j_1}_*\otimes\cdots\otimes x^{j_l}_*\]
    的形式, 其中$A^{i_1\cdots i_k}_{j_1\cdots j_l}\in C^\infty(M)$.
\end{prop}
\begin{proof}
    验证定义~\ref{smooth function 2}~即可.
    对张量来说写出坐标卡的具体表达过于复杂, 我们把这个工作留给读者.
\end{proof}